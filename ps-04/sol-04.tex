\documentclass[11pt]{article}

    \usepackage{listings}
    \usepackage{fancyhdr}
    \usepackage[margin=.8in]{geometry}
    \usepackage{amsmath}
    \usepackage{enumitem}
    
    \linespread{1.3}
    \setlength{\parindent}{0pt}
    \setlength{\tabcolsep}{15pt}
    
    % ===========================================================================
    % Header / Footer
    % ===========================================================================
    \pagestyle{fancy}
    \lhead{\scriptsize  CSC 212: Data Structures and Abstractions - Spring 2018}\chead{}\rhead{\scriptsize Weekly Problem Set \#4}
    \lfoot{}\cfoot{\scriptsize \thepage~of~\pageref{r:lastpage}}\rfoot{}
    \renewcommand{\headrulewidth}{0.3pt}
    \renewcommand{\footrulewidth}{0.3pt}
    
    % ===========================================================================
    % ===========================================================================
    \begin{document}
    \thispagestyle{empty}
    
    % ===========================================================================
    \begin{center}
        {\Large\bf CSC 212: Data Structures and Abstractions}\\
        \medskip
        {\Large\bf Fall 2018}\\
        \medskip
        {\Large\bf University of Rhode Island}\\
        \bigskip
        {\Large\bf Weekly Problem Set \#5}
    \end{center}
    
    Due Thursday 10/11 at the beginning of class. Please turn in neat, and organized, answers hand-written on standard-sized paper \textbf{without any fringe}. At the top of each sheet you hand in, please write your name, and ID.
    
    \begin{enumerate}
        \item Write a recursive function that sums all of the elements of a given array, matching this signature: \begin{verbatim}int sum(int* arr, int n);
        \end{verbatim}
        \begin{verbatim}
            int sum(int* arr, int n) {
                if (n == 0) return 0;
                else {
                    return arr[n-1] + (sum(arr, n-1));
                }
            }
        \end{verbatim}
        \item Rewrite the recursive sum function to only sum odd numbers within the array.
        \begin{verbatim}
            int sum(int* arr, int n) {
                if (n == 0) return 0;
                else if (n % 2) {
                    return arr[n-1] + (sum(arr, n-1));
                } else {
                    return sum(arr, n-1);
                }
            }
        \end{verbatim}
        \item Write a recursive function that can find the minimum of a given array.
        \begin{verbatim}
            int arr_min(int* arr, int n) {
                if (n == 1) return arr[n-1];
                int func_val = arr_min(arr, n-1);
                return arr[n-1] < func_val ? arr[n-1] : func_val;
            }
        \end{verbatim}
        \item Reverse the elements of an array in place. Matching the following function signature: 
    
        \verb|void reverse_array(int* arr, int n);|
        
        \begin{verbatim}
            void reverse_array(int* arr, int n)
            {
                if (n > 1)
                {
                    //swap is not an actual function, have students write it              
                    swap(arr, arr + n - 1);
                    reverse_array(arr + 1, n - 2);
                }
            }
        \end{verbatim}

    \item Write a function to print triangles to \verb|std::cout| that takes three positive integers: $a$, $b$, $c$ as input. The function should print the \verb|+| character $a$ times, then $a+c$ times, then $a+c+c$ times, and so on. This pattern should repeat until the line is $b$ characters long. At that point, the pattern is repeated backwards. For example calling \verb|draw_triangle(4, 7, 1)| will output: (where the dollar symbol is the bash command prompt)
    \begin{verbatim}
        ++++
        +++++
        ++++++
        +++++++
        +++++++
        ++++++
        +++++
        ++++
    \end{verbatim} 
    \pagebreak
    \begin{verbatim}
        void draw_triangle(unsigned a, unsigned b, unsigned c)
        {
            if(a < b)
            {
                for(unsigned i = 0; i < a; i++)
                    std::cout << '+';
                std::cout << '\n';
                draw_triangle(a + c, b, c);
                for(unsigned i = 0; i < a; i++)
                    std::cout << '+';
                std::cout << '\n';
            }
        }
    \end{verbatim} 

    \item Recursively multiply two numbers together, \emph{without using the * operator}. Matching the following function signature:

        \verb|int multiply(int a, int b);|
        
        \begin{verbatim}
        int multiply(int a, int b)
        {
            if(b == 0) return 0;
            else if(b < 0) return a + multiply(a, b + 1);
            else return a + multiply(a, b - 1);
        }
        \end{verbatim} 

    \item Write a function that returns true if a character array is a palindrome and false if it is not. Match the following function signature:
    
        \verb|bool palindrome(char *a, int length);|
    
    \begin{verbatim}
        bool palindrome(char *a, int length)
        {
            if(length <= 1) return true;
            if(a[0] != a[length - 1]) return false;
            return palindrome(a + 1, length - 2);
        }
    \end{verbatim} 
    \item Write a recursive function that returns the nth member of the Fibonacci series, with elements 0 and 1 being 1 and 1 (so the series starts 1, 1, 2, 3, 5, 8, 13, ...). Match the following signature:
    
        \verb|unsigned fibSeries(unsigned n);|
    
    \begin{verbatim}
        unsigned fibSeries(unsigned n)
        {
            if(n == 1 || n == 0) return 1;
            return fibSeries(n - 1) + fibSeries(n - 2);
        }
    \end{verbatim}
    
    \item For both insertion and selection sort, describe if the algorithm is stable and if not give an example array that shows the unstable behavior.
    
    Insertion sort: Stable\\
    Selection sort: Unstable (unless implemented with extra memory or using non-standard insertion method like swapping more than 2 elements per insertion); [1,3,3,2]
    \end{enumerate}
    
    \label{r:lastpage}
    
    \end{document}
        
    