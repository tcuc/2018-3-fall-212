\documentclass[11pt]{article}

    \usepackage{listings}
    \usepackage{fancyhdr}
    \usepackage[margin=.8in]{geometry}
    \usepackage{amsmath}
    \usepackage{enumitem}
    \usepackage{hyperref}
    
    \linespread{1.2}
    \setlength{\parindent}{0pt}
    \setlength{\tabcolsep}{15pt}
    
    % ===========================================================================
    % Header / Footer
    % ===========================================================================
    
    \pagestyle{fancy}
    \lhead{\scriptsize  CSC 212: Data Structures and Abstractions - Fall 2018}\chead{}\rhead{\scriptsize Weekly Problem Set Solutions \#6}
    \lfoot{}\cfoot{\scriptsize \thepage~of~\pageref{r:lastpage}}\rfoot{}
    \renewcommand{\headrulewidth}{0.3pt}
    \renewcommand{\footrulewidth}{0.3pt}
    
    % ===========================================================================
    % ===========================================================================
    \begin{document}
    \thispagestyle{empty}
    
    % ===========================================================================
    \begin{center}
        {\Large\bf CSC 212: Data Structures and Abstractions}\\
        \medskip
        {\Large\bf Fall 2018}\\
        \medskip
        {\Large\bf University of Rhode Island}\\
        \bigskip
        {\Large\bf Weekly Problem Set \#6}
    \end{center}
    
    Due Thursday 11/1 at the beginning of class. Please turn in neat, and organized, answers hand-written on standard-sized paper \textbf{without any fringe}. At the top of each sheet you hand in, please write your name, and ID.
    
    \begin{enumerate}
        \item Implement merge sort. Your function should take an array of integers and the indices of the first and last elements in the list to sort.
        %implementation from https://www.geeksforgeeks.org/merge-sort/
        \begin{verbatim}
// Merges two subarrays of arr[]. 
// First subarray is arr[l..m] 
// Second subarray is arr[m+1..r] 
void merge(int arr[], int l, int m, int r) 
{ 
    int i, j, k; 
    int n1 = m - l + 1; 
    int n2 =  r - m; 
  
    /* create temp arrays */
    int L[n1], R[n2]; 
  
    /* Copy data to temp arrays L[] and R[] */
    for (i = 0; i < n1; i++) 
        L[i] = arr[l + i]; 
    for (j = 0; j < n2; j++) 
        R[j] = arr[m + 1+ j]; 
  
    /* Merge the temp arrays back into arr[l..r]*/
    i = 0; // Initial index of first subarray 
    j = 0; // Initial index of second subarray 
    k = l; // Initial index of merged subarray 
    while (i < n1 && j < n2) 
    { 
        if (L[i] <= R[j]) 
        { 
            arr[k] = L[i]; 
            i++; 
        } 
        else
        { 
            arr[k] = R[j]; 
            j++; 
        } 
        k++; 
    } 
  
    /* Copy the remaining elements of L[], if there 
       are any */
    while (i < n1) 
    { 
        arr[k] = L[i]; 
        i++; 
        k++; 
    } 
  
    /* Copy the remaining elements of R[], if there 
       are any */
    while (j < n2) 
    { 
        arr[k] = R[j]; 
        j++; 
        k++; 
    } 
} 
  
/* l is for left index and r is right index of the 
   sub-array of arr to be sorted */
void mergeSort(int arr[], int l, int r) 
{ 
    if (l < r) 
    { 
        // Same as (l+r)/2, but avoids overflow for 
        // large l and h 
        int m = l+(r-l)/2; 
  
        // Sort first and second halves 
        mergeSort(arr, l, m); 
        mergeSort(arr, m+1, r); 
  
        merge(arr, l, m, r); 
    } 
}
        \end{verbatim}
        \item Implement quicksort. Your function should take an array of integers and the indices of the first and last elements in the list to sort.
        %implementation from https://www.geeksforgeeks.org/quick-sort/
        \begin{verbatim}
void swap(int* a, int* b) 
{ 
    int t = *a; 
    *a = *b; 
    *b = t; 
} 
  
/* This function takes last element as pivot, places 
   the pivot element at its correct position in sorted 
    array, and places all smaller (smaller than pivot) 
   to left of pivot and all greater elements to right 
   of pivot */
int partition (int arr[], int low, int high) 
{ 
    int pivot = arr[high];    // pivot 
    int i = (low - 1);  // Index of smaller element 
  
    for (int j = low; j <= high- 1; j++) 
    { 
        // If current element is smaller than or 
        // equal to pivot 
        if (arr[j] <= pivot) 
        { 
            i++;    // increment index of smaller element 
            swap(&arr[i], &arr[j]); 
        } 
    } 
    swap(&arr[i + 1], &arr[high]); 
    return (i + 1); 
} 
  
/* The main function that implements QuickSort 
 arr[] --> Array to be sorted, 
  low  --> Starting index, 
  high  --> Ending index */
void quickSort(int arr[], int low, int high) 
{ 
    if (low < high) 
    { 
        /* pi is partitioning index, arr[p] is now 
           at right place */
        int pi = partition(arr, low, high); 
  
        // Separately sort elements before 
        // partition and after partition 
        quickSort(arr, low, pi - 1); 
        quickSort(arr, pi + 1, high); 
    } 
}
        \end{verbatim}
        \item Define a recurrence relation for the best case for quicksort. Assume that a partition for an array of size n takes n comparisons.\\
        $T(n) = 2T(n/2) + n$
        \item Solve your recurrence relation.
        $$T(1) = 1; T(n) = 2T(\frac{n}{2}) + n$$
            \begin{equation}
                \begin{split}
                    T(n) & = 2T(\frac{n}{2}) + n \\
                         & = 2(2T(\frac{n}{4}) + \frac{n}{2}) + n \\
                         & = 4T(\frac{n}{4}) + 2n \\
                         & = 8T(\frac{n}{8}) + 3n \\
                         & = \ldots
                \end{split}
            \end{equation}
            This pattern can be written as follows: $$T(n) = 2^kT(\frac{n}{2^k}) + kn$$
            Becoming trivial when $\frac{n}{2^k} = 1$ or $k = \log_{2}n$
            Putting it all together: $$T(n) = nT(1) + n \log_{2}n = n \log_{2}n + n$$
        \item Define a recurrence relation for the worst case for quicksort. Assume that a partition for an array of size n takes n comparisons.\\
        $T(n) = T(1) + T(n - 1) + n = T(n - 1) + n + 1$
        \item Solve your recurrence relation.
        $$T(1) = 1; T(n) = T(n - 1) + n + 1$$
            \begin{equation}
                \begin{split}
                    T(n) & = T(n - 1) + n + 1 \\
                         & = T(n - 2) + n + n + 1 \\
                         & = T(n - 3) + n - 1 + n + n + 1 \\
                         & = T(n - 4) + n - 2 + n - 1 + n + n + 1 \\
                         & = \ldots
                \end{split}
            \end{equation}
            This pattern can be written as follows: $$T(n) = T(n - k) + \sum\limits_{i = n - k + 2}^{n + 1}i$$
            Becoming trivial when $n - k = 1$ or $k = n - 1$
            Putting it all together: $$T(n) = n(n + 1)/2 + n - 1 = n(n + 3)/2 - 1$$
    \end{enumerate}
    \label{r:lastpage}
    \end{document}
        
    