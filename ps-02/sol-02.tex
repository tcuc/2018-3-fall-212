\documentclass[11pt]{article}

\usepackage{listings}
\usepackage{fancyhdr}
\usepackage[margin=.8in]{geometry}
\usepackage{amsmath}
\usepackage{enumitem}

\linespread{1.3}
\setlength{\parindent}{0pt}

% ===========================================================================
% Header / Footer
% ===========================================================================
\pagestyle{fancy}
\lhead{\scriptsize  CSC 212: Data Structures and Abstractions - Fall 2018}\chead{}\rhead{\scriptsize Weekly Problem Set \#1}
\lfoot{}\cfoot{\scriptsize \thepage~of~\pageref{r:lastpage}}\rfoot{}
\renewcommand{\headrulewidth}{0.3pt}
\renewcommand{\footrulewidth}{0.3pt}

% ===========================================================================
% ===========================================================================
\begin{document}
\thispagestyle{empty}

% ===========================================================================
\begin{center}
    {\Large\bf CSC 212: Data Structures and Abstractions}\\
    \medskip
    {\Large\bf Fall 2018}\\
    \medskip
    {\Large\bf University of Rhode Island}\\
    \bigskip
    {\Large\bf Weekly Problem Set \#2}
\end{center}

Due Thursday 9/27 at the beginning of class. Please turn in neat, and organized, answers hand-written on standard-sized paper \textbf{without any fringe}. At the top of each sheet you hand in, please write your name, and ID.

\begin{enumerate}[leftmargin=*]

\item Simplify the following: 
\begin{itemize}
    \item \( \log_2 xy^2 - \log_2 x^2 - 2 \log_2 y \)
    
    \( \log_2 y^2/x - \log_2 y^2 \)
    
    \( \log_2 1/x \)
    \item \( \log_2 (16x^2)^\frac{1}{3} \)
    
    \( \frac{1}{3} (\log_2 (16) + \log_2 (x^2)) \)
    
    \( \frac{4}{3} + \frac{2}{3} \log_2 (x) \)
    \item \( \log_3(9x^4) - \log_3(3x)^2\)
    
    \( \log_3(9) + 4\log_3(x) - 2(\log_3(3) + \log_3(x))\)
    
    \( 2\log_3(3x)\)
\end{itemize}
% https://maths.mq.edu.au/numeracy/web_mums/module2/Worksheet27/module2.pdf

\item Solve for x: \( \log_{2} \frac{x^2}{2} = 5 \)

\( \frac{x^2}{2} = 2^5 \)

\( x^2 = 2^6 \)

\( x = 2^3 \)

\( x = 8 \)

\item Evaluate: \( \sum\limits_{x=0}^3 (5 + \sqrt{4^x}) \)
% https://www.math.ucdavis.edu/~kouba/CalcTwoDIRECTORY/summationdirectory/Summation.html

6 + 7 + 9 + 13 = 35

\item Solve the following: \( \sum\limits_{n=0}^{10} (-n) \)

0 - 1 - 2 - 3 - 4 - 5 - 6 - 7 - 8 - 9 - 10 = -(10)(10 + 1)/2 = -55

\item Prove that: \( \sum\limits_{i=1}^{x} i = \frac{(x + 1)x}{2} \)
% https://opendsa-server.cs.vt.edu/ODSA/Books/Everything/html/Proofs.html#direct-proof

To prove by induction:

Check the base case: n = 1, verify that ((1 + 1) * 1)/2 = 1

State the induction hypothesis: \( \sum\limits_{i=1}^{x-1} i=(((x-1) + 1)(x-1))/2 == ((x-1)x)/2 \)

Thus \( \sum\limits_{i=1}^x i = (\sum\limits_{i=1}^{x-1} i) + x = ((x-1)x)/2 + x = (x^2 - x + 2x)/2 = (x(x+1))/2 \)

\item Rewrite the following expression into its closed form (i.e. without the sigma): \( \sum_{i=1}^4 (2 + i^2) \).

\( (2 + 1^2) + (2 + 2^2) + (2 + 3^2) + (2 + 4^2) \)

\item Based on the given data, please classify each of the following as linear, quadratic, logarithmic, or none of the above. If it is none of the above, try to reason what type of curve it may be.
\begin{itemize}
    \item $f(0) = 4, f(1) = 6, f(2) = 9$
    
    quadratic
    \item $f(0) = 6, f(10) = 8, f(20) = 10$
    
    linear
    \item $f(0) = 80, f(0.1) = 60, f(0.2) = 45$
    
    quadratic
    \item $f(1) = 10, f(10) = 20, f(100) = 30$
    
    logarithmic
    \item $f(0) = 2, f(3) = 12, f(5) = 240$
    
    none (factorial)
    
\end{itemize}
% http://faculty.bard.edu/~mbelk/math141/ExponentialExercises.pdf

\item Rank the following functions by their asymptotic growth rate in ascending order.  In your solution, group those functions that are big-Theta of one another (all $\log$ functions are base 2):
    \begin{equation*}
        \begin{array}{ccccc}
            6 \cdot n\log n & 2^{100} & \log \log n & \log^2 n & 2^{\log n} \\
            2^{2^n} & \lceil\sqrt{n}\rceil & n^{0.01} & 1/n & 4n^{3/2} \\
            4^n & n^3 & n^2\log n & 4^{\log n} & \sqrt{\log n} \\
        \end{array}
    \end{equation*}
    
    \begin{equation*}
            \begin{array}{ccccc}
                1/n & & & & Sub-Constant \\
                2^{100} & & & & Constant \\
                \log \log n & \sqrt{\log n} & \log^2 n & & Logarithmic \\
                n^{0.01} & \lceil\sqrt{n}\rceil & & & Square Root \\
                2^{\log n} & & & & Linear \\
                6 \cdot n\log n & & & & Linearithmic \\
                4n^{3/2} & 4^{\log n} & n^2\log n & n^3 & Polynomial (c>1) \\
                4^n & 2^{2^n} & & & Exponential
            \end{array}           
        \end{equation*}
        Graphed at https://www.desmos.com/calculator/svspzsyq1x

\item For each of the following, give both a big-Oh characterization in terms of $n$, and an exact characterization (count additions and multiplications):
        \begin{enumerate}
            \item
            \begin{verbatim}
            EX: For the following, the big-Oh characterization is: O(n), 
            the exact characterization is n.
            s = 1
            for i = 1 to n do
                s = s * i
            \end{verbatim}
            \item Exactly $4n$, $O(n)$
            \begin{verbatim}
            s = 1
            for i = 1 to 4n do
                s = s * i
            \end{verbatim}
            \item Exactly $n^3$, $O(n^3)$
            \begin{verbatim}
            s = 1
            for i = 1 to n*n*n do
                s = s * i
            \end{verbatim}
            \item Exactly $4n * \frac{4n+1}{2}$, $O(n^2)$
            \begin{verbatim}
            s = 0
            for i = 1 to 4n do
                for j = 1 to i do
                    s = s + i
            \end{verbatim}
            \item Exactly $n^2 * \frac{n^2+1}{2}$, $O(n^4)$
            \begin{verbatim}
            s = 0
            for i = 1 to n*n do
                for j = 1 to i do
                    s = s + i
            \end{verbatim}
            \item Exactly $n*n*n$, $O(n^3)$
            \begin{verbatim}
            s = 1
            for i = 1 to n do
                for j = 1 to n do
                    for k = 1 to n do
                        s = s * i
            \end{verbatim}
        \end{enumerate}

\end{enumerate}

\label{r:lastpage}

\end{document}
    